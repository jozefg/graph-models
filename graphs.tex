\documentclass{amsart}
\usepackage{amsmath, amssymb, stmaryrd}
\usepackage[parfill]{parskip}

\newcommand{\set}[1]{\ensuremath{\mathbf{set}(#1)}}
\newcommand{\unset}[1]{\ensuremath{\mathbf{set}^{-1}(#1)}}
\newcommand{\enc}{\text{enc}}
\newcommand{\dec}{\text{dec}}
\newcommand{\bbrack}[1]{\ensuremath{\llbracket #1 \rrbracket}}

\title{A Simple Model of The Lambda Calculus}
\author{Danny Gratzer}

\begin{document}
\maketitle


Hi folks. Today I'd like to talk about something that came up during
the talks Dana Scott gave this week at CMU. He was discussing a new
way of integrating randomness into models of the lambda calculus, but
the model which he was building this all on was surprisingly simple
and nice. In an attempt to consolidate my notes I'm writing down
everything that I can remember about what are called "graph models" of
the lambda calculus. Hopefully it'll be helpful to someone
else.

Since these are just my notes they come with a huge disclaimer,
they're only correct to the best of my knowledge

\section{Observations about Models of The Lambda Calculus}

If we want to construct a model of the lambda calculus we need to
construct some domain of discourse, $D$ with operations on it letting
us model the critical rules of the lambda calculus. Specifically we
need to be able to simulate one particularly troublesome rule.

\[
    (\lambda x.\ E)E' \equiv [E'/x]E
\]

What makes this rule such a bother is that it forces us construct some
way to convert $d \in D$ into a function upon $D$. Furthermore, in
order to properly interpret lambdas into the model we'll want to find
some way to turn a function on $D$ back into a term on $D$. It
doesn't necessarily need to be an isomorphism, but it certainly needs
to be an embedding. That is, if we have $\enc : (D \to D) \to D$ and

$\dec : D \to (D \to D)$ it really needs to be the case that
$\dec \circ \enc = 1$. It would be even better though to be able to
construct a model where these two collections are isomorphic, meaning
that all functions on our domain are computable functions and all
elements of our domain are computable functions. This is not as easy
as one would hope though, if our domain has 2 distinct elements (it
needs to in order to be a reasonable model) then
$D \to D \cong \mathcal{P}(D)$. Then Cantor pops up and we know that
we won't be able to just build some set so that $\dec \circ \enc = 1$.

This circularity is going to come back up in pretty much any model you
attempt to build and this won't be any exception.

\section{Using Graphs}

In order to construct our model we're going to think of lambda terms
as graphs. They'll pair their input to it's output rendering the term
into one huge stinking table. This is basically how a set theorist
would describe their functions so it's not completely out of no
where. If a lambda term doesn't terminate on a particular input that's
fine, we'll just neglect to add an entry there. This matches our
intuition that two functions are equivalent if they coterminate and
are equivalent on all the cases where they terminate.

So then we can think of how to define $\enc$ and $\dec$.
You should think of $\enc$ as a function squashing a
function into the appropriate look up table.

\begin{align*}
  \enc(F) &= \{(m, n) \mid n = F(m)\}\\
  \dec(d) &= m \mapsto
   \begin{cases}
      n & (m, n) \in d\\
      \bot & \text{otherwise}
   \end{cases}
\end{align*}

There I just mean $\bot$ for undefined, it's notation not some
specific element of our model. This looks so great and yet it's
completely broken. The whole thing starts to unravel as soon as we
ask the question "what are the entries of our tables?" Then we'll
notice that our tables are built of.. tables, made of tables, made of
tables, .. off and off and off to infinity. A little set theory is
enough to convince us that this will lead to paradoxes. In particular
it shouldn't be tricky to show that we couldn't actually realize a set
of such lookup tables.

So now we've reached the main technical challenge, this is the
circularity I've hinted at before; the model we want to define is just
too big to exist. In order to make sure that we don't make the same
mistake twice let's get a little more serious about our math and fix
some concrete set to work with. Specifically, let's look at the
powerset of natural numbers. There's nothing special about the natural
numbers other than them having an infinite number of them.

Our clever idea here is that while we definitely can't encode tables
of tables as sets of numbers, we can encode pairs of natural numbers
as natural numbers. It's not so hard, just
\[
  (n, m) \triangleq 2^n(2m + 1)
\]

("Fun" exercise, convince yourself that this is injective)

Pairs are a good start, but we need more than that, we need a finite
set of pairs. But looking to lisp, we can easily see we can encode
finite sequences with our pairing operation as cons and nil is just
encoded as 0 (our pairing operation never gives 0). And from sequences
there's a simple but woefully non-injective correspondence between
finite sets of pairs and sequence pairs. These can be mapped to
sequences of natural numbers, which are just natural numbers. I'll
use the notation \set{n} to indicate the finite set of natural numbers
$n$ denotes by this process.

So we've G\"odel numbered our way to being able to think of finite
tables as numbers. So, since we're talking about
$\mathcal{P}(\mathbb{N})$, that means we could regard each element
really as an approximation of the structure we were talking about
before where each time we go down a level in the table we go from
infinitely entries at the top to finitely many entries, and then fewer
and fewer until we're completely out. We're forced into a much smaller
model than we had before but it actually exists which is a step up I
suppose.

Now we have to convince ourselves that we can work in such a
restricted model. Let's define a few helpful concepts to help us show
this. Let

\[
  X^* \triangleq \{n \mid \set{n} \subseteq X\}
\]

We then know that
$X^* = \{\unset{F} \mid F \text{ finite} \wedge F \subseteq X\}$. We
can see $X^*$ as a collection of approximations of $X$. It is after
all just an encoding of a set of sets which all describe some finite
fragment of $X$. It's not a terribly efficient way of describing $X$
because there's lots of duplication between approximations but having
all the finite approximations of a set is enough to uniquely determine
it (just union the approximations).

What we do then is define $\dec$ and $\enc$ by piecing together all of
the approximations of relevant sets.
\begin{align*}
  \enc(F) &= \{(n, m) \mid m \in F(\set{n})^*\} \cup \{0\}\\
  \dec(d) &= X \mapsto
    \bigcup \{\set{m} \mid \exists n \in X^*.\ (n, m) \in d\}
\end{align*}

So $\enc$ creates a table mapping a fragment of input to the
appropriate fragment of output. We also tack on $\{0\}$ which is our
G\"odel encoding of $\emptyset$ because ``completely undefined'' is
always a valid approximation of a function, if a poor one. To reverse
the process and get a function out of a lookup table we take our input
and rip it apart into all its valid approximations and piece together
what the lookup table gives us.

The natural question to ask is whether these functions cohere with
each other. Is it the case that $\dec \circ \enc = 1$ for example?
The short answer is no for a variety of reasons. Remember that while
we're encoding everything into a lookup table we can only ever encode
finite entries, infinite sets can't have a G\"odel number. In order
for us to be able prove that $\dec \circ \enc$ could \emph{ever} be an
identity it needs to be possible to determine the behavior of our
function from only the finite cases.

This idea is a driving idea between a lot of Scott's work on lattice
theoretic models. It stems from the observation that computation is
smooth. In the more general sense we can state this property with the
preservation of least upper bounds of directed sets. In our specific
case we can use the more simplistic definition that $F$ is continuous
if
\[
  F(X) = \bigcup \{F(Y) \mid Y \subseteq X \wedge Y \text{ finite}\}
\]

If we apply our trick with G\"odel numbers this is equivalent to
\[
  F(X) = \bigcup \{F(\set{n}) \mid n \in X^*\}
\]

It's not hard to prove that if $F$ is continuous then
$\dec(\enc(F)) = F$. The reverse has some complications, while
$\enc \circ \dec$ will ``morally'' work on the appropriate class of
sets, it won't return the exact same set on the nose. It may return
any of a number of tables which describe the same function in subtly
different ways.

Now finally, we can define a model of the lambda calculus which
satisfies $\alpha$ and $\beta$ conversions.

\begin{align*}
  \bbrack{x}(\rho) &= \rho(x)\\
  \bbrack{M\ N}(\rho) &= \dec(\bbrack{M}(\rho))(\bbrack{N}(\rho))\\
  \bbrack{\lambda x.\ E}(\rho) &= \enc(X \mapsto \bbrack{E}(\rho[x \mapsto X]))
\end{align*}

The unfortunate bit of this model is that it's not extensional. It's
not the case that two lambda terms are equal if and only if they
behave the same when applied. There are many different reasons for
this, but the first and most obvious one is that two lambda terms are
only on the nose equal if they are defined by the same lookup
table. But any given function can be given infinitely many finitely
sized look up tables! This means that even if $\dec$ claims the tables
are the same function they may not be equal as elements of
$\mathcal{P}(\mathbb{N})$. For many applications though this is just
fine, in particular we can still analyze questions of computability
and recursion theory, it's just not a good model to use for proving
some kinds of metatheorems about the lambda calculus.

\section{Conclusion}

This wraps up my presentation of a simple graph model of the lambda
calculus. The key insight into all of this is that computable
functions are continuous. From there it's only natural to take
advantage of the fact that continuous functions are easily encodable
in countable amounts of information which allows us to construct
$\enc$ and $\dec$ so that $\dec \circ \enc = 1$.

Questions I'm left with include

\begin{itemize}
\item This notion of continuity can be tied back to what topology on
  $\mathcal{P}(\mathbb{N})$?\\

  The answer to the question with a few years of hindsight is that,
  yes, it can be. In fact, the topology to which our notion of continuity
  corresponds is called the \emph{Scott topology} and it's precisely
  the topology generated by the basic opens of
  \[
    O_F = \{X \subseteq \mathbb{N} \mid F\ \mathrm{finite} \mathrel{\wedge} F \subseteq X\}
  \]
\item The loss of extensionality is a little disappointing, but it
  seems possible that it would be recoverable if we could quotient our
  set by $\dec$ or something similar. Is it really that easy?\\

  The answer to this question is somewhat less obvious. I would hazard
  a guess that you can recover an extensional model by appropriate
  quotienting but the result is unlikely to be as useful since it
  won't contain the same natural mathematical structure. $D_\infty$
  would be a much more compelling choice if extensionality is truly
  required to show the desired property.
\end{itemize}
\end{document}
